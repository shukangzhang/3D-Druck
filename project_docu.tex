% Link to template used: https://de.overleaf.com/project/5ede5800d140f800011e23ed
%=================================================================
\documentclass[journal,article,submit,moreauthors,pdftex]{Definitions/mdpi} 


%=================================================================
\firstpage{1} 
\makeatletter 
\setcounter{page}{\@firstpage} 
\makeatother
\pubvolume{xx}
\issuenum{1}
\articlenumber{5}
\pubyear{2020}
\copyrightyear{2020}
%\externaleditor{Academic Editor: name}
\history{Received: date; Accepted: date; Published: date}
%\updates{yes} % If there is an update available, un-comment this line


%------------------------------------------------------------------
% The following line should be uncommented if the LaTeX file is uploaded to arXiv.org
%\pdfoutput=1

%=================================================================
% Add packages and commands here. The following packages are loaded in our class file: fontenc, calc, indentfirst, fancyhdr, graphicx, lastpage, ifthen, lineno, float, amsmath, setspace, enumitem, mathpazo, booktabs, titlesec, etoolbox, amsthm, hyphenat, natbib, hyperref, footmisc, geometry, caption, url, mdframed, tabto, soul, multirow, microtype, tikz

% Own added packages
\usepackage{textcomp}
\usepackage{makecell}
\usepackage{ulem}

%=================================================================
%% Please use the following mathematics environments: Theorem, Lemma, Corollary, Proposition, Characterization, Property, Problem, Example, ExamplesandDefinitions, Hypothesis, Remark, Definition, Notation, Assumption
%% For proofs, please use the proof environment (the amsthm package is loaded by the MDPI class).

%=================================================================
% Full title of the paper (Capitalized)
\Title{3D-Druck in der Verfahrenstechnik, Final Project AMIR}

% Author Orchid ID: enter ID or remove command
\newcommand{\orcidauthorA}{0000-0000-000-000X} % Add \orcidA{} behind the author's name
%\newcommand{\orcidauthorB}{0000-0000-000-000X} % Add \orcidB{} behind the author's name

% Authors, for the paper (add full first names)
\Author{Elmar Schiessl, Janick Beck and Shukang Zhang}

% Authors, for metadata in PDF
\AuthorNames{Elmar Schiessl, Janick Beck and Shukang Zhang}

% Affiliations / Addresses (Add [1] after \address if there is only one affiliation.)
\address{%
}

% Contact information of the corresponding author
\corres{}

% Current address and/or shared authorship
\firstnote{} 
\secondnote{}
% The commands \thirdnote{} till \eighthnote{} are available for further notes

%\simplesumm{} % Simple summary

%\conference{} % An extended version of a conference paper

% Abstract (Do not insert blank lines, i.e. \\) 
\abstract{}

% Keywords
\keyword{3D-Printing, FLM, AMIR, Mixing Reactor}

% The fields PACS, MSC, and JEL may be left empty or commented out if not applicable
%\PACS{J0101}
%\MSC{}
%\JEL{}



%%%%%%%%%%%%%%%%%%%%%%%%%%%%%%%%%%%%%%%%%%
\begin{document}
%%%%%%%%%%%%%%%%%%%%%%%%%%%%%%%%%%%%%%%%%%

%%%%%%%%%%%%%%%%%%%%%%%%%%%%%%%%%%%%%%%%%%
\section{Introduction}


 
%%%%%%%%%%%%%%%%%%%%%%%%%%%%%%%%%%%%%%%%%%
\section{Results}


%%%%%%%%%%%%%%%%%%%%%%%%%%%%%%%%%%%%%%%%%%
\section{Discussion}


%%%%%%%%%%%%%%%%%%%%%%%%%%%%%%%%%%%%%%%%%%
\section{Materials and Methods}


%%%%%%%%%%%%%%%%%%%%%%%%%%%%%%%%%%%%%%%%%%
\section{Conclusions}



%%%%%%%%%%%%%%%%%%%%%%%%%%%%%%%%%%%%%%%%%%
\vspace{6pt} 

%%%%%%%%%%%%%%%%%%%%%%%%%%%%%%%%%%%%%%%%%%
%% optional
%\supplementary{The following are available online at \linksupplementary{s1}, Figure S1: title, Table S1: title, Video S1: title.}

% Only for the journal Methods and Protocols:
% If you wish to submit a video article, please do so with any other supplementary material.
% \supplementary{The following are available at \linksupplementary{s1}, Figure S1: title, Table S1: title, Video S1: title. A supporting video article is available at doi: link.}

%%%%%%%%%%%%%%%%%%%%%%%%%%%%%%%%%%%%%%%%%%
\authorcontributions{}

%%%%%%%%%%%%%%%%%%%%%%%%%%%%%%%%%%%%%%%%%%
\funding{This research received no external funding}

%%%%%%%%%%%%%%%%%%%%%%%%%%%%%%%%%%%%%%%%%%
\acknowledgments{.}

%%%%%%%%%%%%%%%%%%%%%%%%%%%%%%%%%%%%%%%%%%
\conflictsofinterest{} 

%%%%%%%%%%%%%%%%%%%%%%%%%%%%%%%%%%%%%%%%%%
%% optional
\abbreviations{The following abbreviations are used in this manuscript:\\

\noindent 
\begin{tabular}{@{}ll}
MDPI & Multidisciplinary Digital Publishing Institute\\
DOAJ & Directory of open access journals\\
TLA & Three letter acronym\\
LD & linear dichroism
\end{tabular}}

%%%%%%%%%%%%%%%%%%%%%%%%%%%%%%%%%%%%%%%%%%
%% optional
\appendixtitles{yes} %Leave argument "no" if all appendix headings stay EMPTY (then no dot is printed after "Appendix A"). If the appendix sections contain a heading then change the argument to "yes".
\appendix
\section{Requirement Specification}
 \begin{table}[H]
\caption{Requirement specification list for the heat exchanger}
\centering
%% \tablesize{} %% You can specify the fontsize here, e.g., \tablesize{\footnotesize}. If commented out \small will be used.
\begin{tabular}{llllll}
\toprule
\textbf{\makecell{Obligatory\\or Desirable}} & \textbf{Category}	& \textbf{Description}	& \textbf{Value} & \textbf{\makecell{Person\\Responsible}} & \textbf{Last Changed} \\
\midrule
%    &   &   &  \\
Obligatory & Performance		& volume flow rate			& greater 0.5 m$^3$/h   & Scherf &  2020-06-23 \\
Obligatory    & Performance  & heat distribution  & radial and evenly   & Scherf & 2020-06-23 \\
Obligatory    & Performance  & heating water flowing trough  & from 10 °C to 80 °C   & Scherf & 2020-06-23 \\
Obligatory  & Performance   & low pressure drop &   & Scherf & 2020-06-23 \\
Obligatory	& Material    	& material temperature resistance			& between 0 °C and 90 °C   & Group A & 2020-06-23 \\
Obligatory	& Material    	& material water solvability  & unsolvable in water   & Group A & 2020-06-23 \\
Obligatory    & Material  & electrical conductivity  & greater $10^6$ S/m   & Group A & 2020-06-23 \\
Obligatory	& Manufacturing    	& manufacturing process  & additive manufacturing   & Scherf & 2020-06-23 \\
Desirable	& Geometry    	& customizability of model & model is parameterized   & Group A & 2020-06-23 \\
Obligatory	& Geometry    	& outer Shape  & cylindrical   & Group A & 2020-06-23 \\
Obligatory	& Geometry    	& outer diameter  &  94 mm   & Scherf & 2020-06-23 \\
Desirable	& Geometry    	& length  & up to 300 mm   & Scherf & 2020-06-23 \\
\sout{Obligatory}    & \sout{Geometry}  & \sout{outer wall}  & \sout{closed}   &   & 2020-06-23 \\
\sout{Desirable}    & \sout{Geometry}  & \sout{outer wall thickness} & \sout{smaller 5 mm}   &  & 2020-06-23 \\
Obligatory    & Geometry  & water flow direction  & fixed direction   & Scherf & 2020-06-23 \\

%%% Besprechung 09.06.2020 Notizen (+ markierte Punkte wurden in der Anforderungliste uebernommen):
% + Low pressure drop
% + Höhere Temperaturtoleranz, da das bauteil heißer wird als das Wasser#
% Effizienzgrad? Anderes Paper als Benchmark, oder verschiedene Versionen des eigenen Modells verbessern
% + Parametrisierbarkeit könnte Anforderung sein
% Statische Mischer als Anfangsstrukturen für Energieeintrag und Durchmischung
% + Länge des Zylinders: Festgelegt durch Versuchsaufbau und 3D-Drucker
% + Länge durch Versuchsaufbau: 1 m tube length - Ein- und Ablaufzone
% + Länge festgelegt: Bis 300 mm
% Druck dürfen wir selbst festlegen
% Befestigung im Rohr: Spaltmaß berücksichtigen, Fügemaße abschätzen, Befestigung durch Klemmringe (Eigentlich Sprengring innerhalb des Rohrs)
% + Es gibt feste Flußrichtung
% Anforderungen an das System stellen und auch an das Bauteil
% + Verantwortlichkeit fehlt in Anforderungsliste (3 Foliensatz)
% Aus welchem Material besteht das Rohr? Plexiglass PMMA, bis 90 °C Temperaturstabil
% Kleiners Rohr (1 Zoll) auch möglich für erste Versuche
% Anforderungsliste an Experimente.
% Leistung der Spule, Berchnen aus Anforderungen
% Anforderung, Wie radiale Wärme berechnen?

%%% Besprechung 16.06.2020
% Reduzierung der Durchflussmenge bei Liveversuch auf realistischere Werte
% Person responsible ist die Person für die die Anforderung erstellt wird
% Kupferspule ringsherum
% Performance Erhitzung des Wassers nur bis 80 °C
% Zylinder muss keine Hülle haben
% Bachelorarbeit Restriktionsgerechte Strukturoptimierung von additiv gefertigten Strukturreaktoren mit STar-ccm+, Lars Grobelny
% Datum der Änderungen mit in Anforderungsliste, da es benötigt wird
% Wir kriegen Fotos vom Versuchsstand
% Wir kriegen Simulationsmodell der Bachelorarbeit
% Anforderungsliste soll in den Anhang des Dokuments
% Zeitplanung bis spätestens nächste Woche vorlegen, mit Angabe, wer für was verantwortlich ist (gun chart?)
% Teile dürfen abbrechen von unserer Tube
% Funktionsmodell können wir schon anfangen mit z. B. Brainstorming was
% Funktionsmodell sollte in ca. 5 h zu machen sein
% Bis nächste Woche ein diskussionsfähiges Funktionsmodell vorlegen
% Projektplanung der nächsten Wochen vorstellen
% Nächste Besprechung Mittwoch 24.06.2020 um 10 Uhr 
\bottomrule
\end{tabular}
\end{table}

%%%%%%%%%%%%%%%%%%%%%%%%%%%%%%%%%%%%%%%%%%
% Citations and References in Supplementary files are permitted provided that they also appear in the reference list here. 

%=====================================
% References, variant A: internal bibliography
%=====================================
\reftitle{References}
\begin{thebibliography}{999}
% Reference 1
\bibitem[Author1(year)]{ref-journal}
Author1, T. The title of the cited article. {\em Journal Abbreviation} {\bf 2008}, {\em 10}, 142--149.
% Reference 2
\bibitem[Author2(year)]{ref-book}
Author2, L. The title of the cited contribution. In {\em The Book Title}; Editor1, F., Editor2, A., Eds.; Publishing House: City, Country, 2007; pp. 32--58.
\end{thebibliography}

% The following MDPI journals use author-date citation: Arts, Econometrics, Economies, Genealogy, Humanities, IJFS, JRFM, Laws, Religions, Risks, Social Sciences. For those journals, please follow the formatting guidelines on http://www.mdpi.com/authors/references
% To cite two works by the same author: \citeauthor{ref-journal-1a} (\citeyear{ref-journal-1a}, \citeyear{ref-journal-1b}). This produces: Whittaker (1967, 1975)
% To cite two works by the same author with specific pages: \citeauthor{ref-journal-3a} (\citeyear{ref-journal-3a}, p. 328; \citeyear{ref-journal-3b}, p.475). This produces: Wong (1999, p. 328; 2000, p. 475)

%=====================================
% References, variant B: external bibliography
%=====================================
%\externalbibliography{yes}
%\bibliography{your_external_BibTeX_file}

%%%%%%%%%%%%%%%%%%%%%%%%%%%%%%%%%%%%%%%%%%
%% optional
%\sampleavailability{Samples of the compounds ...... are available from the authors.}

%% for journal Sci
%\reviewreports{\\
%Reviewer 1 comments and authors’ response\\
%Reviewer 2 comments and authors’ response\\
%Reviewer 3 comments and authors’ response
%}

%%%%%%%%%%%%%%%%%%%%%%%%%%%%%%%%%%%%%%%%%%
\end{document}

